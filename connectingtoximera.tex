\documentclass{ximera}

\title{Connecting to Ximera}

\begin{document}

%\begin{abstract}
%Instructions for setting up a repository containing course materials.
%\end{abstract}

\maketitle

This activity guides you to connect your \textbf{GitHub} repository to the \textbf{Ximera server}.

\begin{enumerate}

\item Go to \link[\sf Ximera]{http://ximera.osu.edu} and sign in with your Google account.

\item In order to connect \textbf{Ximera} to your \textbf{GitHub}: 
Click first your username at the top right corner and in the dropdown menu click \textbf{Profile} 
and then \textbf{Connect GitHub} at the bottom of the page.

\item Click \textbf{Settings} botton on your \texttt{github.com} repository and then \textbf{Webhooks \& Services}. 
 Click the \textbf{Add webhook} botton and continue by confirming with your password. 
 At the \textbf{Webhooks/Add webhook} page, type \texttt{http://497a6980.ngrok.com/github} in the \textbf{Payload URL} field 
 and \texttt{8mi0tsrje9n3asPu86XC198G1XSdZj} into the \textbf{Secret} field. Finally push the green Add webhook button at the bottom of the window.

\item\textbf{Step 4.} Create a new TeX file in your repository. You can do this in the GitHub web interface 
(See the online documentation at \link[\sf Creating files on GitHub]{https://github.com/blog/1327-creating-files-on-github}.) 
Call this file ``\texttt{hello.tex}." Copy the following text into your new document:

\begin{verbatim}
\documentclass{ximera}
\begin{document}
\title{Hello World!}
\begin{abstract}
This is a basic ximera example.
\end{abstract}
\maketitle

\begin{question}
What is the airspeed velocity of an unladen swallow? $\answer{3}$
\end{question}
\end{document}
\end{verbatim}

\item Saving your work in the GitHub repository is called \textbf{committing};  
after pasting the above code, scroll down and click the \textbf{Commit new file} button.

\item Saving your work in the GitHub repository is called \textbf{committing}. 
Click on the word ``branch" (or ``branches"). 
In the master branch row you should see either a yellow circle or a green checkmark. 
\textcolor{yellow}{Yellow} means that ximera has received your tex document and is processing it; 
\textcolor{green}{green} means it is done. 
A \textcolor{red}{red X} here means there is a problem; see Troubleshooting for info.

\item Once you see the green checkmark in the branch listing, 
you can view your document at \texttt{http://497a6980.ngrok.com/course/USERNAME/REPONAME/master/filename}, 
where USERNAME and REPONAME are the names you took note of earlier.
(Note. By adding ``.tex" to the URL above you can view the tex source and see a compilation log; 
this is useful in case there is a problem.)

\end{enumerate}

Congratulations! You have successfully create a \textbf{Ximera activity}. 

\end{itemize}


\end{document}
