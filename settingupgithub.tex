\documentclass{ximera}

\def\sectionautorefname~#1\null{\S#1\null}
\def\subsectionautorefname~#1\null{\S#1\null}
\def\itemautorefname~#1\null{#1\null}

\title{Setting up a GitHub Repository}

\begin{document}

%\begin{abstract}
%Instructions for setting up a repository containing course materials.
%\end{abstract}

\maketitle

%Assume now that you are familiar with \LaTeX\ and 

\begin{itemize}

\item\textbf{Step 1.} Go to \link[\tt githup.com]{http://github.com} and follow their instructions to create a GitHub account. 
% (\emph{Remark.} As long as you leave your repository public, you can create as mnay repositories as you can.)

\item\textbf{Step 2.} Start a new repository by clicking the \emph{$+$ New repository} button. 
Then follow the steps below \textbf{BEFORE} clicking the \emph{Create repository} button.

\begin{itemize}

\item Name the repository. (Remember this name.)
\item Leave the repository \textbf{Public}.
\item Check \textbf{Initialize this repository with a README}.
\item Add a license (We recommend \emph{Creative Commons Zero})

\end{itemize}

After completing all the steps above, click \textbf{Create repository}.

Congratulations! You now have a GitHub account and have created a repository that will store your files to be used by Ximera.

\end{itemize}


\end{document}
